\documentclass{article}
\usepackage{amssymb,amsmath,amsfonts,amsthm}
% Set builder notation
\newcommand{\set}[2]{
	\{\ #1 \mid #2\ \}
}

% ?=
\newcommand{\qeq}{\stackrel{?}{=}}

% def=
\newcommand{\defeq}{\stackrel{\text{def}}{=}}

% Definition block with label
\newcommand{\DEFINITION}[1]{
  \label{def-#1}
  {\noindent \bf Definition #1}
}

% Theorem block with label
\newcommand{\THEOREM}[1]{
  \label{theorem-#1}
  {\noindent \bf #1 Theorem}
}

% Generator: <a>
\newcommand{\gen}[1]{
  \langle #1 \rangle
}

% Modulo: (mod 3)
\newcommand{\modulo}[1]{
  \ (\textrm{mod}\ #1)
}


\newcommand{\congr}[3]{
  #1 \equiv #2\ (\textrm{mod}\ #3)
}
\begin{document}

\section{Number Theory and Abstract Algebra}

This takes up roughly 15\% of the test, but is partially involved in other
questions. This is a common mix-in, to make a problem harder.

\section{Divisibility}

\subsection{Quick Rules for Factoring}

\begin{itemize}
  \item by 2, iff the last digit is divisible by 2
  \item by 3, iff the sum of the digits is divisible by 3
  \item by 4, iff the last \emph{two} digitis is divisible by 4
  \item by 5, iff the last digit is 0 or 5
  \item by 8, iff the last \emph{three} digits are divisible by 8
  \item by 9, iff the sum of the digits is divisible by 9
\end{itemize}

\subsection{Division Algorithm}

If $a, b \in \mathbb{Z}^+$, then $\exists q, r \in \mathbb{Z} : b = qa + r$


\subsection{Primes}

$ \forall \in \mathbb{Z}^+, \exists \text{prime} p : k < p < 2k $

$ \sum{k=1}^n{\frac{1}{p_k}} $ divierges where $p_k$ is the k-th prime

\subsection{GCD and LCD}

Greatest-common-division and least-common-denominator come up in group size and
and various algorithmic problems.

\pagebreak
\subsubsection{GCD}

...

For any integers $a, b$, we can write:
  \begin{align*}
    a   & = (p_1)^{a_1} (p_2)^{a_2} ... (p_k)^{a_k}     \\
    b   & = (p_1)^{b_1} (p_2)^{b_2} ... (p_k)^{b_k}     \\
    m_i & \defeq min(a_i, b_i)                          \\
    M_i & \defeq max(a_i, b_i)                          \\
    gcd(a, b) & \defeq (p_1)^{m_1} (p_2)^{m_2} (\dots) (p_k)^{m_k} \\
    lcM(a, b) & \defeq (p_1)^{M_1} (p_2)^{M_2} (\dots) (p_k)^{M_k} \\
  \end{align*}

As a result:
  $$ gcd(a, b) \cdot lcm(a, b) = a \cdot b $$

\subsubsection{Euclidean Algorithm}

Algorithmically determine the greatest common divisor between two numbers. This
divides the larger number by the smaller and continually checks whether it did
it evenly. When $r = 0$ we know $ r | b_k, b_{k-1}, b_{k-2}, ..., b_0 $ and that
it must be the largest such number that does it.

\begin{verbatim}
  def gcd (a, b):
    if b < a: (a, b) = (b, a)

    r = b % a

    while r != 0:
      # Note: b = a * m + r
      b = a
      a = r
      m = b // a
      r = b % a

    return a
\end{verbatim}

\subsubsection{Diophantine Equation $ax +by = c$}

The Simple Linear Diophantine Equations $ax + by = c$ has solutions of the form:\\
  $ x = x_1 + \frac{b}{d} t $\\
  $ y = y_1 - \frac{a}{d} t $

  Finding $x_1$ and $y_1$ is the only real hard part. We know that we can \emph{always} find solutions of the form: \\
  $ a x_0 + b y_0 = d $ where $d = gcd(a, b)$

  After solving this equation, we multiply both sides of the equation by $\frac{c}{d}$ to get:
  $ a \frac{c}{d} x_0 + b \frac{c}{d} y_0 = d \frac{c}{d} $

  Define: \\
  $ x_1 = \frac{c}{d} x_0 $ \\
  $ y_1 = \frac{c}{d} y_0 $ \\

  Then plug in these values of $x_1$ and $y_1$ to attain solutions: \\
  $ x = \frac{c}{d} x_0 + \frac{b}{d} t $
  $ y = \frac{c}{d} y_0 - \frac{a}{d} t $

  As an aside, note the negative in y's value. This seems to relate to the Euclidean algorithm.


  \pagebreak
  \subsection{Congruences}

  Rules for congruences:
  \begin{itemize}
    \item $\congr{a}{b}{n}$ and $\congr{b}{c}{n}$, then $\congr{a}{c}{n}$

    \item
      $\congr{a}{b}{n}$, then $\forall c \in \mathbb{Z}$
      \begin{itemize}
        \item $\congr{a \pm c}{b \pm c}{n}$
        \item $\congr{a c}{b c}{n}$
      \end{itemize}

    \item
      If $\congr{a_1}{b_1}{n}$ and $\congr{a_2}{b_2}{n}$, then:
      \begin{itemize}
        \item $\congr{a_1 \pm a_2}{b_1 \pm b_2}{n}$
        \item $\congr{a_1 a_2}{b_1 b_2}{n}$
      \end{itemize}

    \item
      $\forall\ 0 < c \in \mathbb{Z}$, the following are equivalent:
      \begin{itemize}
        \item $\congr{a}{b}{n}$
        \item $\congr{a}{b + n}{cn}$
        \item $\congr{a}{b + 2n}{cn}$
        \item \dots
        \item $\congr{a}{b + (c-1)n}{cn}$
      \end{itemize}

    \item
      If $\congr{ab}{ac}{n}$, then:
      \begin{itemize}
        \item $\congr{b}{c}{n}$ if $d\defeq$ gcd($a$, $n$) $= 1$
        \item $\congr{b}{c}{\frac{n}{d}}$ if $d\defeq$ gcd($a$, $n$) $> 1$
      \end{itemize}

    \item
      The linear congruence equation $\congr{ax}{b}{n}$ has a solution iff gcd($a$, $n$) $ = d | b$ and:
      \begin{itemize}
        \item if $d=1$, then the solution is unique mod $n$
        \item if $d>1$, then the solution is unique mod $\frac{n}{d}$
      \end{itemize}
  \end{itemize}

  \subsection{Congruence Equation $\congr{ax}{b}{n}$}

  Note that Linear Congruence Equations are \emph{equivalent} to
  Simple, Linear Diophantine Equations.

  Solving these requires applying rules 1-7 of the previous section. They are pretty arduous.

  \section{Abstract Algebra}

\end{document}
